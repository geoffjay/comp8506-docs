\section{Project Description}\label{sec:desc}

  \subsection{Background}\label{sec:desc-bg}

    \subsubsection{Student}\label{sec:desc-bg-student}

      Geoff Johnson is a student of the BCIT Computer Systems Bachelor's degree.
      Prior to this he obtained a diploma of Computer Systems Technology from
      Camosun college in Victoria in 2001, as well as a diploma in Robotics and
      Automation Technology from BCIT in 2006.

      For the past 10 years Geoff has worked in Burnaby for the fluid mechanics
      research and development company Coanda where he is responsible for managing
      the IT operations, as well as managing and contributing to control system
      software development. This software is used to interface with industrial
      instrumentation, perform automated feedback control of plant processes,
      and capture data to log files at upwards of 100,000 samples per second.

    \subsubsection{Company}\label{sec:desc-bg-company}

      Coanda is an engineering research and development company specializing in
      industrial fluid dynamics and related technologies. They have been in
      business for close to 20 years providing services in process flow modelling,
      design optimization, and custom instrumentation to name a few. For more
      information their website is \url{https://www.coanda.ca}.

    \subsubsection{History}\label{sec:desc-bg-project}

      The project that was proposed, and is the subject of this report, was the
      reproduction of an existing piece of software used to measure and control
      physical processes. The software that this project seeks to improve upon
      is comprised of two software projects, a library of objects that abstract
      components of a data acquisition and control system, and a GUI application
      that functions as an operator panel.

      The first is a library of objects that abstracts the data acquisition
      hardware as various types of measurement channels, it handles logging
      to both CSV files as well as databases, feedback control loop
      calculations using the PID (Proportional Integral Derivative) equation,
      and handles the creation of object hierarchy through a class builder
      using XML as the input. This library, libcld, is an open source project
      hosted in a git repository at \url{https://github.com/geoffjay/libcld}.

      The second piece of software is a user facing GUI application developed for
      the GNOME system. It interfaces with libcld, hereafter referred to as
      CLD (Control, Logging, and Data Acquisition), using a single context to
      provide the data necessary to update the interface view. This application
      has also been released as open source and is hosted in a git repository
      at \url{https://github.com/coanda/dactl}.

  \subsection{Proposed Solution}\label{sec:desc-soln}

    Performing all of the work in a single application for data acquisition,
    feedback control, and data logging is open to a variety of issues. Doing so
    limits hardware scalability, places critical operations onto systems running
    desktop software, and ties high reliability functions like plant control in
    the same process as the front end that the end user operates. These are just
    a few of the more obvious problems.

    The system being proposed in this document is one that implements all of
    the hardware access and logging functionality as micro-services that run as
    processes detached from the plant operators that interact with the GUI
    application. This server software will use an existing library to interface
    with the data acquisition hardware as well as perform the control and
    logging tasks, this aspect of the software is a reimplementation of work
    that already exists in an application that is currently in use.

    New functionality that is planned is the definition and minimal
    implementation of a REST API, a messaging system for streaming data, and a
    configuration definition to allow for the reconfiguration of the daemon. The
    REST API is meant to provide a means for clients to make simple requests of
    the system like read a single measurement value or change a property value
    of one of the internal objects. ZeroMQ is to be used to implement the
    messaging system, it has several high level features including publish and
    subscribe socket types that can be used for a streaming data system, a
    request/reply structure for a higher performance version of the REST API,
    and a structure for bridging communications to allow for the reconfiguration
    of the server as a proxy or slave node in a distributed system.
