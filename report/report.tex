%
% OpenDCS Proposal
%
% Author: Geoff Johnson
%
% To compile: pdflatex --shell-escape --synctex=1 --interaction=nonstopmode report.tex
%

\documentclass[11pt]{article}

\usepackage[titles]{tocloft}
\usepackage{verbatim}
\usepackage[pdftex]{graphics,graphicx}
\usepackage[export]{adjustbox}
\usepackage{booktabs}
\usepackage{pdfpages}
\usepackage{float}
\usepackage{amssymb}
\usepackage{mathtools}
\usepackage{biblatex}
\usepackage{svg}
\usepackage{tikz}
\usepackage{listings,lstautogobble}
\usepackage[tablegrid]{vhistory}
\usetikzlibrary{arrows}
\bibliography{references}

\usepackage{hyperref}
\hypersetup{%
  pdfauthor={Geoff Johnson},
  pdftitle={OpenDCS Project Report},
  pdfsubject={Project Report},
  pdfkeywords={OpenDCS Project Report},
  colorlinks=true,
  linkcolor=black,
  urlcolor=blue
}

% Change listing setup.
\definecolor{sol-basic}{RGB}{101,123,131} % Base0
\definecolor{sol-comment}{RGB}{147,161,161} % Base1
\definecolor{sol-keyword}{RGB}{133,153,0} % Green
\definecolor{sol-string}{RGB}{203,75,22} % Orange
\definecolor{sol-emph}{RGB}{211,54,130} % Magenta
\definecolor{sol-background}{RGB}{253,246,227} % Base3

\lstset{
  language=xml,
  backgroundcolor=\color{sol-background},
  identifierstyle=\ttfamily,
  basicstyle=\footnotesize\ttfamily\color{sol-basic},
  keywordstyle=\color{sol-keyword},
  commentstyle=\color{sol-comment},
  stringstyle=\color{sol-string},
  emph={dcs,objects,object,property},
  emphstyle={\color{sol-emph}},
  tabsize=4,
  aboveskip={1.5\baselineskip},
  columns=fixed,
  inputencoding=utf8,
  extendedchars=true,
  breaklines=true,
  prebreak = \raisebox{0ex}[0ex][0ex]{\ensuremath{\hookleftarrow}},
  frame=single,
  showtabs=false,
  showspaces=false,
  showstringspaces=false,
  numbers=left,
  numberstyle=\tiny,
  breakatwhitespace=true,
  title=\lstname,
  literate={│}{{\textSFxi}}1 {└}{{\textSFii}}1 {├}{{\textSFviii}}1 {─}{{\textSFx}}1,
  autogobble=true,
}

% Packages used for appendices.
\usepackage{appendix}
\usepackage{listings}
\usepackage{color}
\definecolor{light-gray}{gray}{0.95}
\definecolor{listinggray}{gray}{0.9}
\definecolor{lbcolor}{rgb}{0.9,0.9,0.9}
\definecolor{light-blue}{rgb}{0.6,0.720,0.85}

% The default margins are too wide all the way around, reset them.
\setlength{\topmargin}{-.5in}
\setlength{\textheight}{9in}
\setlength{\oddsidemargin}{0in}
\setlength{\textwidth}{6.5in}

% Change paragraph formatting.
\setlength{\parindent}{0pt}
\setlength{\parskip}{2ex}

\begin{document}
\nocite{*}

  \title{%
    OpenDCS $-$ An Open Distributed Control System\vspace{2em}
  }

  \author{%
    Geoff Johnson \vspace{0.5em} \\
    geoff.jay@gmail.com \vspace{0.5em} \\
    A00533481 \vspace{0.5em} \\
    COMP8045 \vspace{0.5em}
  }

  \maketitle
  \thispagestyle{empty}
  \newpage
  \mbox{}
  \thispagestyle{empty}

  \newpage
  \addtocounter{page}{-1}
  \pagenumbering{roman}
  \tableofcontents
  \listoffigures
  \listoftables
  \lstlistoflistings

  \newpage
  \pagenumbering{arabic}

  % XXX fill in or omit sections as needed, for now just blasting ideas

  \section{Project Description}\label{sec:desc}

    \subsection{Background}\label{sec:desc-bg}

  \section{Requirements}\label{sec:req}

    The stakeholders that have been identified from Coanda are:

    \begin{table}[H]
      \centering
      \begin{tabular}{l p{6cm} p{6cm}}
        \toprule
        Name & Organization Role & Project Role \\ [0.5ex]
        \midrule
        Geoff Johnson & Software Developer        & Project Lead \\
        Stephen Roy   & Software Developer        & Contributor \\
        Scott Webster & Research Scientist        & Requirements Contributor \\
        Thomas Depew  & Instrumentation Scientist & Requirements Contributor \\
        Bernie LeSage & Chief Operations Officer  & Project Reviewer \\
        \bottomrule
      \end{tabular}
      \caption{Project Stakeholders}\label{tab:stakeholders}
    \end{table}

    \subsection{Raw List from Project Stakeholders}\label{sec:req-sh}

      \begin{table}[H]
        \centering
        \begin{tabular}{l p{11cm} p{3cm}}
          \toprule
          Number & Description & Priority (1-5)\newline 1 is highest \\ [0.5ex]
          \midrule
          1 & Devices must be loaded as plugins & 1 \\
          \bottomrule
        \end{tabular}
        \caption{Initial Requirements List}\label{tab:requirements}
      \end{table}

    \subsection{Analysis}\label{sec:req-analyze}

    \subsection{System Requirements Specifications}\label{sec:req-srs}

      \subsubsection{Core Library}\label{sec:req-srs-core}

      \subsubsection{Data Acquisition Library}\label{sec:req-srs-daq}

      \subsubsection{Data Logging Library}\label{sec:req-srs-log}

      \subsubsection{Process Control Library}\label{sec:req-srs-ctl}

      \subsubsection{Device Plugins}\label{sec:req-srs-dev-plug}

      \subsubsection{Logging Plugins}\label{sec:req-srs-log-plug}

      \subsubsection{Control Plugins}\label{sec:req-srs-ctl-plug}

      \subsubsection{Data Acquisition Daemon}\label{sec:req-srs-daqd}

      \subsubsection{Data Logging Daemon}\label{sec:req-srs-logd}

      \subsubsection{Process Control Daemon}\label{sec:req-srs-ctld}

  \section{Design}\label{sec:dsg}

  \section{Modeling}\label{sec:mod}

    \subsection{Use Cases}\label{sec:mod-use}

      % good example of use case in "Bootstrap grids day 2" http://www.bossable.com/577/mean-stack-bootstrap/

  \section{Development}\label{sec:dev}

    \subsection{Autotools Build Systems}\label{sec:dev-ac}

  \section{Installation}\label{sec:inst}

    \subsection{Environment}\label{sec:inst-env}

    \subsection{Dependencies}\label{sec:inst-dep}

    \subsection{Containerized Components}\label{sec:inst-cont}

  \section{Configuration}\label{sec:cfg}

    For all OpenDCS objects that are the type \emph{Object} configuration can be
    represented for \emph{property} as data, \emph{id} and \emph{type} as
    attributes, and additionally \emph{property} should have a \emph{name}
    attribute. For all that are the type \emph{Container} will have as an
    addition an array of \emph{Object} nodes.

    \subsection{XML Format}\label{sec:cfg-xml}

      Object:

      \begin{lstlisting}[caption={Object Configuration in XML},label={lst:cfg-xml-obj}]
        <core:object id="ds0" type="data-series">
          <property name="buffer-size">100</property>
          <property name="stride">10</property>
        </core:object>
      \end{lstlisting}

      Container:

      \begin{lstlisting}[caption={Container Configuration in XML},label={lst:cfg-xml-ctr}]
        <core:object id="" type="">
          <property name=""></property>
          <core:object id="" type="" />
          <core:object id="" type="" />
        </core:object>
      \end{lstlisting}

      Nested Containers:

      \begin{lstlisting}[caption={Nested Container Configuration in XML},label={lst:cfg-xml-ctr}]
        <core:object id="" type="">
          <property name=""></property>
          <core:object id="" type="">
            <property name=""></property>
            <core:object id="" type="" />
            <core:object id="" type="" />
          </core:object>
        </core:object>
      \end{lstlisting}

    \subsection{JSON Format}\label{sec:cfg-json}

      Object:

      \begin{lstlisting}[caption={Object Configuration in JSON},label={lst:cfg-json-obj}]
        {
          "id": "ds0",
          "type": "data-series",
          "properties": [
            { "name": "buffer-size", "value": 100 },
            { "name": "stride", "value": 10 }
          ]
        }
      \end{lstlisting}

      Container:

      \begin{lstlisting}[caption={Container Configuration in JSON},label={lst:cfg-json-ctr}]
        {
          "id": "ctr0",
          "type": "container",
          "properties": [
            { "name": "sample", "value": "sample" }
          ],
          "objects": [
            { "id": "", "type": "" },
            { "id": "", "type": "" }
          ]
        }
      \end{lstlisting}

      Nested Containers:

      \begin{lstlisting}[caption={Nested Container Configuration in JSON},label={lst:cfg-json-ctr}]
        {
          "id": "ctr0",
          "type": "container",
          "properties": [
            { "name": "sample", "value": "sample" }
          ],
          "objects": [
            {
              "id": "samp0",
              "type": "sample",
              "properties": [
                { "name": "sample", "value": "sample" }
              ],
              "objects": [
                { "id": "", "type": "" },
                { "id": "", "type": "" }
              ]
            },
          ]
        }
      \end{lstlisting}

  \section{Testing}\label{sec:test}

  \section{Results}\label{sec:res}

  % Insert a list of references that were cited.
  %\newpage
  %\printbibliography%

  % Appendices
  \newpage
  \addappheadtotoc%
  \appendix
  \appendixpage%

  % Left over from a previous proposal document, edit later as needed.
  \section{Glossary of Abbreviations and Terms}\label{app:glossary}

    Abbreviations and terms used within this document for computing and
    software related topics are given in Table~\ref{tab:gloss:sw}, and in
    Table~\ref{tab:gloss:hw} for hardware and data acquisition related topics.

    \begin{table}[H]
      \centering
      \begin{tabular}{l p{12cm}}
        \toprule
        Term & Definition/Explanation \\ [0.5ex]
        \midrule
        API & Application Programming interface \\
        Client & Refers to the software application that communications with a daemon \\
        Container & Operating system level virtualization similar to chroot jails \\
        CRC & Cyclic Redundancy Check, an error checking mechanism \\
        CRUD & Create, Read, Update, Delete \\
        Daemon & Common term used to refer to a server application in a Linux system \\
        DevOps & A practice emphasizing collaboration between developers and IT professionals \\
        Docker & Container software applications used for DevOps and SysOps \\
        GLib & Standard set of Linux system libraries for the GNOME window manager \\
        GNU & A collection of applications, libraries, and developer tools \\
        GNOME & Open source dekstop environment software used with Linux systems \\
        GObject & A C type library to gain object oriented style features \\
        GUI & Graphical User Interface \\
        OSS & Open Source Software \\
        Perl & A high level programming language good for rapid development \\
        REST & Representational State Transfer \\
        RS232 & Serial communication devices that has been ubiquitous for decades \\
        SDD & Software Specification Design Document \\
        SRS & Software Requirements Specification \\
        SysOps & System operator of a multi-user computing system \\
        UML & Unified Modeling Language \\
        Vala & An object oriented programming language that uses GObject types \\
        valadoc & A documentation standard and tool for creating API descriptions \\
        XML & Extensible markup language, common for use in messaging systems \\
        \bottomrule
      \end{tabular}
      \caption{Glossary of Software Terms}\label{tab:gloss:sw}
    \end{table}

    \begin{table}[H]
      \centering
      \begin{tabular}{l p{12cm}}
        \toprule
        Term & Definition/Explanation \\ [0.5ex]
        \midrule
        Comedi & Open source hardware drivers for Control and Measurement Devices \\
        Container & Operating system level virtualization similar to chroot jails \\
        Data Acquisition & The act of gathering data from a real world process \\
        DAQ & Acronym for Data Acquisition \\
        Plant & Term commonly used for industrial control systems \\
        Watchdog & Standard concept for monitoring a vital systems heartbeat \\
        \bottomrule
      \end{tabular}
      \caption{Glossary of Hardware Terms}\label{tab:gloss:hw}
    \end{table}

  \newpage

  \section{Websites Referenced}\label{app:websites}

    A list of websites that have been referenced in this document. These have
    been presented here instead of wherever the hyperlink has been made
    because in some cases very long URLs in printed documents can be less useful
    than an active link in a digital one.

    \begin{table}[H]
      \centering
      \begin{tabular}{l p{10cm}}
        \toprule
        Name & URL \\ [0.5ex]
        \midrule
        Coanda     & https://www.coanda.ca \\
        CRUD       & https://en.wikipedia.org/wiki/Create,\_read,\_update\_and\_delete \\
        dactl      & https://github.com/coanda/dactl \\
        libcld     & https://github.com/geoffjay/libcld \\
        Markdown   & https://en.wikipedia.org/wiki/Markdown \\
        Unit tests & https://github.com/geoffjay/libcld/tree/master/tests \\
        Valadoc    & http://valadoc.org \\
        \bottomrule
      \end{tabular}
      \caption{URL Reference List}\label{tab:websites}
    \end{table}

  \newpage

  \section{Sample Configurations}\label{app:configurations}

    Configuration files (XML and JSON) that will have been referenced throughout
    this document.

    \begin{lstlisting}[caption={Complete Core Sample},label={lst:full-core}]
      <?xml version="1.0"?>
      <dcs xmlns:core="libdcs-core">
        <property name="app">Sample</property>
        <core:objects>
          <core:object id="0" type="test"/>
        </core:objects>
      </dcs>
    \end{lstlisting}

    \begin{lstlisting}[caption={Complete DAQ Sample},label={lst:full-daq}]
      <?xml version="1.0"?>
      <dcs xmlns:core="urn:libdcs-core" xmlns:daq="urn:libdcs-daq">
        <property name="app">Sample</property>
        <core:objects>
          <core:object id="0" type="test"/>
        </core:objects>
      </dcs>
    \end{lstlisting}

    \begin{lstlisting}[caption={Complete Log Sample},label={lst:full-log}]
      <?xml version="1.0"?>
      <dcs xmlns:core="urn:libdcs-core" xmlns:log="urn:libdcs-log">
        <property name="app">Sample</property>
        <core:objects>
          <core:object id="0" type="test"/>
        </core:objects>
      </dcs>
    \end{lstlisting}

  \newpage

  \section{Document Change Log}\label{app:changelog}

    The table below serves to track the key revisions made to this document for
    change control purposes.

    \begin{versionhistory}
      \vhEntry{0.1}{2016-12-28}{Geoff Johnson}{Create initial report (pages: all)}
      %\vhEntry{}{}{}{}{}
      %\vhEntry{}{}{}{}{}
    \end{versionhistory}

\end{document}
