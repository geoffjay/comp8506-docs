\section{Configuration}\label{sec:cfg}

  For all OpenDCS objects that are the type \emph{Object} configuration can be
  represented for \emph{property} as data, \emph{id} and \emph{type} as
  attributes, and additionally \emph{property} should have a \emph{name}
  attribute. For all that are the type \emph{Container} will have as an
  addition an array of \emph{Object} nodes.

  \subsection{XML Format}\label{sec:cfg-xml}

    The previous project that OpenDCS was forked from, dactl, was developed to
    use XML from the beginning which is why this format will continue to be
    allowed as a configuration method.

    Service Schema:

    \begin{lstlisting}[language=XML,caption={OpenDCS Service Schema},label={lst:cfg-xml-schema}]
      <?xml version="1.0" encoding="UTF-8"?>
      <xs:schema xmlns:xs="http://www.w3.org/2001/XMLSchema"
                 elementFormDefault="qualified">
        <xs:element name="dcs">
          <xs:complexType>
            <xs:sequence>
              <xs:element ref="property" maxOccurs="unbounded"/>
              <xs:element ref="object" minOccurs="0" maxOccurs="unbounded"/>
            </xs:sequence>
          </xs:complexType>
        </xs:element>
        <xs:element name="object">
          <xs:complexType>
            <xs:sequence>
              <xs:element ref="property" maxOccurs="unbounded"/>
              <xs:element ref="reference" maxOccurs="unbounded"/>
              <xs:element ref="object" minOccurs="0" maxOccurs="unbounded"/>
            </xs:sequence>
            <xs:attribute name="id" type="xs:string" use="required"/>
            <xs:attribute name="type" type="xs:string" use="required"/>
          </xs:complexType>
        </xs:element>
        <xs:element name="property">
          <xs:complexType mixed="true">
            <xs:attribute name="name" type="xs:string" use="optional"/>
          </xs:complexType>
        </xs:element>
        <xs:element name="reference">
          <xs:complexType>
            <xs:attribute name="path" type="xs:string" use="optional"/>
          </xs:complexType>
        </xs:element>
      </xs:schema>
    \end{lstlisting}

    Object:

    \begin{lstlisting}[caption={Object Configuration in XML},label={lst:cfg-xml-obj}]
      <core:object id="ds0" type="data-series">
        <property name="buffer-size">100</property>
        <property name="stride">10</property>
      </core:object>
    \end{lstlisting}

    Container:

    \begin{lstlisting}[caption={Container Configuration in XML},label={lst:cfg-xml-ctr}]
      <core:object id="" type="">
        <property name=""></property>
        <core:object id="" type="" />
        <core:object id="" type="" />
      </core:object>
    \end{lstlisting}

    Nested Containers:

    \begin{lstlisting}[caption={Nested Container Configuration in XML},label={lst:cfg-xml-nest-ctr}]
      <core:object id="" type="">
        <property name=""></property>
        <core:object id="" type="">
          <property name=""></property>
          <core:object id="" type="" />
          <core:object id="" type="" />
        </core:object>
      </core:object>
    \end{lstlisting}

  \subsection{JSON Format}\label{sec:cfg-json}

    Object:

    \begin{lstlisting}[caption={Object Configuration in JSON},label={lst:cfg-json-obj}]
      {
        "id": "ds0",
        "type": "data-series",
        "properties": [
          { "name": "buffer-size", "value": 100 },
          { "name": "stride", "value": 10 }
        ]
      }
    \end{lstlisting}

    Container:

    \begin{lstlisting}[caption={Container Configuration in JSON},label={lst:cfg-json-ctr}]
      {
        "id": "ctr0",
        "type": "container",
        "properties": [
          { "name": "sample", "value": "sample" }
        ],
        "objects": [
          { "id": "", "type": "" },
          { "id": "", "type": "" }
        ]
      }
    \end{lstlisting}

    Nested Containers:

    \begin{lstlisting}[caption={Nested Container Configuration in JSON},label={lst:cfg-json-nest-ctr}]
      {
        "id": "ctr0",
        "type": "container",
        "properties": [
          { "name": "sample", "value": "sample" }
        ],
        "objects": [
          {
            "id": "samp0",
            "type": "sample",
            "properties": [
              { "name": "sample", "value": "sample" }
            ],
            "objects": [
              { "id": "", "type": "" },
              { "id": "", "type": "" }
            ]
          },
        ]
      }
    \end{lstlisting}
