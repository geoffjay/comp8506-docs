\section{Requirements}\label{sec:req}

  The stakeholders that have been identified from Coanda are:

  \begin{table}[H]
    \centering
    \begin{tabular}{l p{6cm} p{6cm}}
      \toprule
      Name & Organization Role & Project Role \\ [0.5ex]
      \midrule
      Geoff Johnson & Software Developer        & Project Lead \\
      Stephen Roy   & Software Developer        & Contributor \\
      Scott Webster & Research Scientist        & Requirements Contributor \\
      Thomas Depew  & Instrumentation Scientist & Requirements Contributor \\
      Bernie LeSage & Chief Operations Officer  & Project Reviewer \\
      \bottomrule
    \end{tabular}
    \caption{Project Stakeholders}\label{tab:stakeholders}
  \end{table}

  \subsection{Raw List from Project Stakeholders}\label{sec:req-sh}

    During a meeting with project contributors at the beginning of the
    requirements gathering process the question ``What would you like to be
    able to do with the system'' was posed. A summary of the responses is
    given below in Table~\ref{tab:requirements} where each may or may not
    be directly applicable to this stage of the project. For instance elements
    of the user interface that are of interest are not a component of this
    work, instead what will be attempted with these is an analysis of how the
    system can be structured to support that development in the future.

    \begin{table}[H]
      \centering
      \begin{tabular}{l p{11cm} p{3cm}}
        \toprule
        Number & Description & Priority (1-5)\newline 1 is highest \\ [0.5ex]
        \midrule
         1 & Devices must be loaded as plugins & 1 \\
         2 & Want to run any service on a RaspberryPI & 3 \\
         3 & Like to run HMI on Windows systems & 5 \\
         4 & Messages from services should be test-able using curl & 2 \\
         5 & Data stream monitoring should be possible using tcpdump-like utility & 2 \\
         6 & Message compression and decompression using msgpack is desired & 3 \\
         7 & Configuration for all components needs to be consistent and similar to existing format used by libcld and dactl & 1 \\
         8 & About/help systems for components should show versions and patches in use & 2 \\
         9 & Must be able to develop plugins in Python as well as C/Vala & 1 \\
        10 & Plugin skeletons should be possible to generate from templates & 3 \\
        11 & Must be able to run any individual component on the same or separate host & 2 \\
        12 & Plotting data in some format is a must & 2 \\
        13 & The data logger data must be query-able from that service over a given date and time range & 1 \\
        \bottomrule
      \end{tabular}
      \caption{Initial Requirements List}\label{tab:requirements}
    \end{table}

    \begin{table}[H]
      \centering
      \begin{tabular}{l p{11cm} p{3cm}}
        \toprule
        Number & Description & Priority (1-5)\newline 1 is highest \\ [0.5ex]
        \midrule
        14 & Time synchronization between the services that run on different computers should be possible & 5 \\
        15 & Some method of broadcasting to the services should be provided for messages that apply to the system as a whole & 3 \\
        16 & Must be able to acquire data from different devices and device types & 1 \\
        17 & Logging data to plain text is a must, to special binary formats is a nice to have & 1 \\
        18 & Must be able to perform feedback control of actuators using acquired data & 1 \\
        19 & Documentation needs to be available to provide end users the know how to launch the services & 3 \\
        20 & Documentation needs to be available to describe how to configure the services & 3 \\
        21 & The build process should allow you to compile and install individual components as well as the system as a whole & 4 \\
        22 & Data needs to be available over networked computers & 2 \\
        23 & A well structured messaging format needs to be defined for other services and consumers to adhere to & 2 \\
        24 & ZeroMQ should be used as the message queue system & 1 \\
        25 & Services should also provide a RESTful API to make their information available to other software regardless of the language used to develop it & 2 \\
        26 & Data acquisition devices need to be developed as plugins that are loaded into the DAQ service & 1 \\
        27 & Data logging backends like XML or databases need to be developed as plugins that are loaded into the data logging service & 1 \\
        28 & Feedback controllers need to be developed as plugins that are loaded into the control service & 1 \\
        29 & Would be nice to be able to test and possibly distribute the different services as Dockerized applications & 4 \\
        30 & Should be able to build and load the various plugins for the system using documentation provided & 3 \\
        31 & Should be able to install any system component by following the documentation provided & 1 \\
        32 & Should be able to configure any system component by following the documentation provided & 2 \\
        33 & Should be able to test any system component by following the documentation provided & 3 \\
        \bottomrule
      \end{tabular}
      \caption{Initial Requirements List Continued}\label{tab:requirements-cont}
    \end{table}

  \subsection{Analysis}\label{sec:req-analyze}

  \subsection{System Requirements Specifications}\label{sec:req-srs}

    The system requirements for OpenDCS are described in this section, it
    contains information about the functionality of the system as a whole.
    OpenDCS is a platform that has the potential to integrate with electrical
    and mechanical systems so it is necessary to comment on the software
    interactions with such systems, but due to the nature of the design it
    would be impossible to enforce strict rules governing external systems
    that are mechanical, electrical and software.

    This section contains all of the functional and quality requirements of
    the system. A detailed description of the system and all its features is
    given. The following is a list of the abbreviations and acronyms used to
    document these requirements.

    \begin{itemize}
      \setlength{\itemindent}{.5in}
      \itemsep .15em
      \item[ID:] requirement identifier
      \item[TAG:] short form of name
      \item[DESC:] description
      \item[RAT:] rationale
      \item[DEP:] dependencies
      \item[GIST:] rough summary
      \item[SCALE:] the unit for the req to be measured in
      \item[METER:] the way in which the measurement will be performed, can specify test
      \item[PAST:] benchmark typ. for products dev'd in past
      \item[REC:] (record) benchmark represents best value achieved for meter
      \item[MUST:] minimal goal value
      \item[PLAN:] desired goal value
      \item[WISH:] ideal goal value
    \end{itemize}

    \subsubsection{External Interface Requirements}\label{sec:req-srs-ext}

      %\paragraph{User Interfaces}

      \paragraph{Hardware Interfaces}

        \begin{itemize}
          \setlength{\itemindent}{.5in}
          \itemsep .15em
          \item[ID:] HWIR-0001
          \item[TAG:] RS-232 Serial Port
          \item[DESC:] A serial port data acquisition device should be able
            to connect to the system.
          \item[RAT:] Plugin device development will make use of serial port
            core functionality.
        \end{itemize}

        \begin{itemize}
          \setlength{\itemindent}{.5in}
          \itemsep .15em
          \item[ID:] HWIR-0002
          \item[TAG:] Ethernet
          \item[DESC:] The system shall communicate using 10BASE-T,
            100BASE-TX and 1000BASE-T.
          \item[RAT:] The physical system components connect to the network
            using Ethernet.
        \end{itemize}

        \begin{itemize}
          \setlength{\itemindent}{.5in}
          \itemsep .15em
          \item[ID:] HWIR-0003
          \item[TAG:] Number of Ethernet Ports
          \item[DESC:] The number of ports required shall depend on how the
            system configuration for a particular application and whether it
            needs to connect to remote hardware.
          \item[RAT:] Multiple physical hosts can be used to configure a
            system of services.
        \end{itemize}

        \begin{itemize}
          \setlength{\itemindent}{.5in}
          \itemsep .15em
          \item[ID:] HWIR-0004
          \item[TAG:] Processor speed
          \item[DESC:] The system components shall each have a 400 MHz or
            faster processor.
          \item[RAT:] The software requires a reasonably fast processor to
            handle the data and run a user interface.
        \end{itemize}

        \begin{itemize}
          \setlength{\itemindent}{.5in}
          \itemsep .15em
          \item[ID:] HWIR-0005
          \item[TAG:] Display
          \item[DESC:] A 1920 x 1080 (16:9) monitor should be used.
          \item[RAT:] A good user experience that represents a reasonable
            amount of data necessarily requires a display with sufficient
            resolution.
        \end{itemize}

      \paragraph{Software Interfaces}

        \begin{itemize}
          \setlength{\itemindent}{.5in}
          \itemsep .15em
          \item[ID:] SWIR-0001
          \item[TAG:] Operating System
          \item[DESC:] The software architecture shall be 64-bit GNU/Linux
            (e.g. Fedora 25)
          \item[RAT:] The system is intended to be compiled with modern
            software dependencies available in newer Linux distributions.
        \end{itemize}

      \paragraph{Communications Interfaces}

        \begin{itemize}
          \setlength{\itemindent}{.5in}
          \itemsep .15em
          \item[ID:] CIR-0001
          \item[TAG:] ZeroMQ
          \item[DESC:] The system shall use ZeroMQ for publishing and
            subscribing to service data, and to request and reply to
            individual components.
          \item[RAT:] A messaging system is required for easily extending the
            service oriented architecture, ZeroMQ offers a lot of options for
            future concepts.
        \end{itemize}

        \begin{itemize}
          \setlength{\itemindent}{.5in}
          \itemsep .15em
          \item[ID:] CIR-0002
          \item[TAG:] REST
          \item[DESC:] Each service of the system shall provide a RESTful API
            for data access to external components that need/prefer to use
            HTTP requests.
          \item[RAT:] Having this API available to application developers
            simplifies the time required to do simple tasks.
        \end{itemize}

    \subsubsection{Functional Requirements}\label{sec:req-srs-func}

      \paragraph{Common Requirements}

        \begin{itemize}
          \setlength{\itemindent}{.5in}
          \itemsep .15em
          \item[ID:] FR-0001
          \item[TAG:] Connectivity
          \item[DESC:] Requirements that enable communication between and are
            common to all system components that shall be met.
          \item[RAT:] Interoperability of the system components requires a
            common messaging queue.
          \item[DEP:] FR-0002, FR-0003, FR-0004
        \end{itemize}

        \begin{itemize}
          \setlength{\itemindent}{.5in}
          \itemsep .15em
          \item[ID:] FR-0002
          \item[TAG:] Publishing Socket
          \item[DESC:] A configurable ZeroMQ publishing socket shall be
            available.
          \item[RAT:] A service shall be able to publish data on a ZeroMQ
            socket.
          \item[DEP:] none
        \end{itemize}

        \begin{itemize}
          \setlength{\itemindent}{.5in}
          \itemsep .15em
          \item[ID:] FR-0003
          \item[TAG:] Subscribing Socket
          \item[DESC:] A configurable ZeroMQ subscribing socket shall be
            available.
          \item[RAT:] A service can subscribe to published data from other
            system components.
          \item[DEP:] none
        \end{itemize}

        \begin{itemize}
          \setlength{\itemindent}{.5in}
          \itemsep .15em
          \item[ID:] FR-0004
          \item[TAG:] Request/Reply Socket
          \item[DESC:] A ZeroMQ request/reply socket shall be available.
          \item[RAT:] A service can request and receive data from a connected
            system component.
          \item[DEP:] none
        \end{itemize}

        \begin{itemize}
          \setlength{\itemindent}{.5in}
          \itemsep .15em
          \item[ID:] FR-0005
          \item[TAG:] Device Plugin System
          \item[DESC:] A libpeas plugin system must be available to load data
            acquisition devices.
          \item[RAT:] The data acquisition service will get its data to
            transmit from device plugins that are loaded at runtime.
          \item[DEP:] none
        \end{itemize}

        \begin{itemize}
          \setlength{\itemindent}{.5in}
          \itemsep .15em
          \item[ID:] FR-0006
          \item[TAG:] Logging Backend Plugin System
          \item[DESC:] A libpeas plugin system must be available to load data
            logging backends.
          \item[RAT:] The data logging service will log the data it receives
            to backend plugins that are loaded at runtime.
          \item[DEP:] none
        \end{itemize}

        \begin{itemize}
          \setlength{\itemindent}{.5in}
          \itemsep .15em
          \item[ID:] FR-0007
          \item[TAG:] Feedback Controller Plugin System
          \item[DESC:] A libpeas plugin system must be available to load
            feedback controllers.
          \item[RAT:] The feedback control service will use data that it
            receives to perform feedback control using plugins that are
            loaded at runtime.
          \item[DEP:] none
        \end{itemize}

      \paragraph{Data Acquisition}

        \begin{itemize}
          \setlength{\itemindent}{.5in}
          \itemsep .15em
          \item[ID:] DSFR-0001
          \item[TAG:] Data acquisition common system requirements
          \item[DESC:] The common service requirements shall be met.
          \item[RAT:] There are common requirements of a DAQ system that
            should be met.
          \item[DEP:] FR-0001
        \end{itemize}

        \begin{itemize}
          \setlength{\itemindent}{.5in}
          \itemsep .15em
          \item[ID:] DSFR-0002
          \item[TAG:] Publish Data Messages
          \item[DESC:] Data messages shall be published on the socket
            containing channel measurements, timestamps, and broadcast
            status/warnings.
          \item[RAT:] ZeroMQ publishing sockets provide required functionality.
          \item[DEP:] FR-0002
        \end{itemize}

        \begin{itemize}
          \setlength{\itemindent}{.5in}
          \itemsep .15em
          \item[ID:] DSFR-0003
          \item[TAG:] Subscribe to Data Messages
          \item[DESC:] The data acquisition system component shall be able to
            subscribe to published messages.
          \item[RAT:] ZeroMQ subscribing sockets provide required functionality.
          \item[DEP:] FR-0003
        \end{itemize}

        \begin{itemize}
          \setlength{\itemindent}{.5in}
          \itemsep .15em
          \item[ID:] DSFR-0004
          \item[TAG:] Start
          \item[DESC:] The data acquisition system component shall start
            publishing data in response to a request to do so if it has not
            already started.
          \item[RAT:] A DAQ service should be capable of starting on demand.
          \item[DEP:] FR-0002
        \end{itemize}

        \begin{itemize}
          \setlength{\itemindent}{.5in}
          \itemsep .15em
          \item[ID:] DSFR-0005
          \item[TAG:] Stop
          \item[DESC:] The data acquisition system component shall stop
            publishing data in response to a request to do so.
          \item[RAT:] A DAQ service should be capable of stopping on demand.
          \item[DEP:] FR-0002
        \end{itemize}

        \begin{itemize}
          \setlength{\itemindent}{.5in}
          \itemsep .15em
          \item[ID:] DSFR-0006
          \item[TAG:] Automatic Start
          \item[DESC:] The data acquisition system component shall start
            publishing data immediately.
          \item[RAT:] A DAQ service should be configurable to start necessary
            components of the system at launch.
          \item[DEP:] FR-0002
        \end{itemize}

      \paragraph{Data Logging}

        \begin{itemize}
          \setlength{\itemindent}{.5in}
          \itemsep .15em
          \item[ID:] LSFR-0001
          \item[TAG:] Data logging common system requirements
          \item[DESC:] The common service requirements shall be met.
          \item[RAT:] There are common requirements of a data logging system
            that should be met.
          \item[DEP:] FR-0001
        \end{itemize}

        \begin{itemize}
          \setlength{\itemindent}{.5in}
          \itemsep .15em
          \item[ID:] LSFR-0002
          \item[TAG:] Publish backend malfunction warning
          \item[DESC:] Data messages shall be published as a warning when its
            back end data storage is not functioning.
          \item[RAT:] To allow the subscribing application interface to alert
            the user.
          \item[DEP:] FR-0002
        \end{itemize}

        \begin{itemize}
          \setlength{\itemindent}{.5in}
          \itemsep .15em
          \item[ID:] LSFR-0003
          \item[TAG:] Subscribe to published data from data acquisition system.
          \item[DESC:] The data logging system component shall be able to
            subscribe to published messages.
          \item[RAT:] ZeroMQ subscribing sockets provide required functionality.
          \item[DEP:] FR-0003
        \end{itemize}

        \begin{itemize}
          \setlength{\itemindent}{.5in}
          \itemsep .15em
          \item[ID:] LSFR-0004
          \item[TAG:] Start
          \item[DESC:] The data logging system component shall start
            subscribing to data in response to a request to do so if it has not
            already started.
          \item[RAT:] A log service should be capable of starting on demand.
          \item[DEP:] FR-0002
        \end{itemize}

        \begin{itemize}
          \setlength{\itemindent}{.5in}
          \itemsep .15em
          \item[ID:] LSFR-0005
          \item[TAG:] Stop
          \item[DESC:] The data logging system component shall stop
            subscribing to data in response to a request to do so.
          \item[RAT:] A log service should be capable of stopping on demand.
          \item[DEP:] FR-0002
        \end{itemize}

        \begin{itemize}
          \setlength{\itemindent}{.5in}
          \itemsep .15em
          \item[ID:] LSFR-0006
          \item[TAG:] Automatic Start
          \item[DESC:] The data logging system component shall start
            publishing data immediately.
          \item[RAT:] A log service should be configurable to start necessary
            components of the system at launch.
          \item[DEP:] FR-0002
        \end{itemize}

      \paragraph{Feedback Control}

        \begin{itemize}
          \setlength{\itemindent}{.5in}
          \itemsep .15em
          \item[ID:] CSFR-0001
          \item[TAG:] Feedback control common system requirements
          \item[DESC:] The common service requirements shall be met.
          \item[RAT:] There are common requirements of a feedback control system
            that should be met.
          \item[DEP:] FR-0001
        \end{itemize}

        \begin{itemize}
          \setlength{\itemindent}{.5in}
          \itemsep .15em
          \item[ID:] CSFR-0002
          \item[TAG:] Subscribe to input data
          \item[DESC:] The feedback control system will subscribe to
            published data from the data acquisition system and filter out
            the channel measurement data that it needs.
          \item[RAT:] Input data is required for its calculations.
          \item[DEP:] FR-0003
        \end{itemize}

        \begin{itemize}
          \setlength{\itemindent}{.5in}
          \itemsep .15em
          \item[ID:] CSFR-0003
          \item[TAG:] Publish output data
          \item[DESC:] The feedback control system shall publish the data
            resultant of it calculations.
          \item[RAT:] Output data is required to control process variable
            outputs.
          \item[DEP:] FR-0002
        \end{itemize}

        \begin{itemize}
          \setlength{\itemindent}{.5in}
          \itemsep .15em
          \item[ID:] CSFR-0004
          \item[TAG:] Perform Feedback Control
          \item[DESC:] Feedback control of hardware performed by the service
            must be possible.
          \item[RAT:] Hardware control is fundamental to a distributed
            control system.
          \item[DEP:] FR-0002, FR-0003, CSFR-0002, CSFR-0003
        \end{itemize}

        \begin{itemize}
          \setlength{\itemindent}{.5in}
          \itemsep .15em
          \item[ID:] CSFR-0005
          \item[TAG:] Start
          \item[DESC:] The feedback control system component shall start
            publishing and subscribing to data in response to a request to do
            so if it has not already started.
          \item[RAT:] A feedback control service should be capable of starting
            on demand.
          \item[DEP:] FR-0002
        \end{itemize}

        \begin{itemize}
          \setlength{\itemindent}{.5in}
          \itemsep .15em
          \item[ID:] CSFR-0006
          \item[TAG:] Stop
          \item[DESC:] The feedback control system component shall stop
            publishing and subscribing to data in response to a request to do
            so.
          \item[RAT:] A feedback control service should be capable of stopping
            on demand.
          \item[DEP:] FR-0002
        \end{itemize}

        \begin{itemize}
          \setlength{\itemindent}{.5in}
          \itemsep .15em
          \item[ID:] CSFR-0007
          \item[TAG:] Automatic Start
          \item[DESC:] The feedback system component shall start
            publishing data immediately.
          \item[RAT:] A feedback control service should be configurable to start
            necessary components of the system at launch.
          \item[DEP:] FR-0002
        \end{itemize}

    \subsubsection{Non-Functional Requirements}\label{sec:req-srs-non-func}

      \paragraph{Performance}

        The requirements in this section provide a detailed specification of
        the user interaction with the software and measurements placed on the
        system performance.

        Usage of the data log query feature

        \begin{itemize}
          \setlength{\itemindent}{.5in}
          \itemsep .15em
          \item[ID:] QR-0001
          \item[GIST:] Data log query response time.
          \item[SCALE:] seconds
          \item[METER:] Measurements taken from the time a query request was
            sent until the response was received and reported as the difference.
          \item[MUST:] 100 ms
          \item[PLAN:] 10 ms
          \item[WISH:] 500 us
        \end{itemize}

        Configuration modification request response time

        \begin{itemize}
          \setlength{\itemindent}{.5in}
          \itemsep .15em
          \item[ID:] QR-0004
          \item[GIST:] Response time taken for a successful configuration
            modification using the REST API.
          \item[SCALE:] seconds
          \item[METER:] Measurements taken from the time a query request was
            sent until the response was received and reported as the difference.
          \item[MUST:] 100 ms
          \item[PLAN:] 10 ms
          \item[WISH:] 500 us
        \end{itemize}

        Configuration information request response time

        \begin{itemize}
          \setlength{\itemindent}{.5in}
          \itemsep .15em
          \item[ID:] QR-0005
          \item[GIST:] Response time taken for a request to configuration
            using the REST API.
          \item[SCALE:] seconds
          \item[METER:] Measurements taken from the time a query request was
            sent until the response was received and reported as the difference.
          \item[MUST:] 100 ms
          \item[PLAN:] 10 ms
          \item[WISH:] 500 us
        \end{itemize}

        Sample rate capacity of data acquisition service

        \begin{itemize}
          \setlength{\itemindent}{.5in}
          \itemsep .15em
          \item[ID:] QR-0006
          \item[GIST:] Bulk messages sent count as delivered by the data
            acquisition service.
          \item[SCALE:] MPS
          \item[METER:] Measurements taken by publishing a set number of
            messages sent by the publisher and received by the subscriber and
            reported as the total received.
          \item[MUST:] 100 MPS
          \item[PLAN:] 100 MPS
          \item[WISH:] 1000 MPS
          \item[MPS:] Messages per second
        \end{itemize}

        Sample rate capacity of data logging service

        \begin{itemize}
          \setlength{\itemindent}{.5in}
          \itemsep .15em
          \item[ID:] QR-0007
          \item[GIST:] Bulk messages received count as seen by the data logging
            service.
          \item[SCALE:] MPS
          \item[METER:] Measurements taken by publishing a set number of
            messages sent by the publisher and received by the subscriber and
            reported as the total received.
          \item[MUST:] 100 MPS
          \item[PLAN:] 100 MPS
          \item[WISH:] 1000 MPS
          \item[MPS:] Messages per second
        \end{itemize}

        Sample rate capacity of feedback control service

        \begin{itemize}
          \setlength{\itemindent}{.5in}
          \itemsep .15em
          \item[ID:] QR-0008
          \item[GIST:] Bulk messages sent and received as delivered and seen
            by the feedback control service.
          \item[SCALE:] MPS
          \item[METER:] Measurements taken by publishing a set number of
            messages sent by the publisher and received by the subscriber and
            reported as the total received.
          \item[MUST:] 100 MPS
          \item[PLAN:] 100 MPS
          \item[WISH:] 1000 MPS
          \item[MPS:] Messages per second
        \end{itemize}

        System dependability of the data acquisition service

        \begin{itemize}
          \setlength{\itemindent}{.5in}
          \itemsep .15em
          \item[ID:] QR-0009
          \item[GIST:] The fault tolerance of the system.
          \item[SCALE:] If the system loses its connection to the building
            network, another connected component, or a malformed message, the
            user should be informed.
          \item[METER:] Measurements obtained from 100 hours of usage during
            testing.
          \item[MUST:] 99\% of the time
          \item[PLAN:] 100\% of the time
          \item[WISH:] 100\% of the time
        \end{itemize}

        System dependability of the data logging service

        \begin{itemize}
          \setlength{\itemindent}{.5in}
          \itemsep .15em
          \item[ID:] QR-0010
          \item[GIST:] The fault tolerance of the system.
          \item[SCALE:] If the system loses its connection to the building
            network, another connected component, or a malformed message, the
            user should be informed.
          \item[METER:] Measurements obtained from 100 hours of usage during
            testing.
          \item[MUST:] 99\% of the time
          \item[PLAN:] 100\% of the time
          \item[WISH:] 100\% of the time
        \end{itemize}

        System dependability of the feedback control service

        \begin{itemize}
          \setlength{\itemindent}{.5in}
          \itemsep .15em
          \item[ID:] QR-0011
          \item[GIST:] The fault tolerance of the system.
          \item[SCALE:] If the system loses its connection to the building
            network, another connected component, or a malformed message, the
            user should be informed.
          \item[METER:] Measurements obtained from 100 hours of usage during
            testing.
          \item[MUST:] 99\% of the time
          \item[PLAN:] 100\% of the time
          \item[WISH:] 100\% of the time
        \end{itemize}

      \paragraph{Design Constraints}

        This section includes the design constraints on the software caused by
        the hardware.

        Network bandwidth

        \begin{itemize}
          \setlength{\itemindent}{.5in}
          \itemsep .15em
          \item[ID:] QR-0012
          \item[GIST:] Ethernet network bandwidth capacity.
          \item[SCALE:] Mbps
          \item[METER:] Measure the total throughput of the network performance
            by flooding with message traffic and monitor using \texttt{tcpdump}.
          \item[MUST:] 100 Mbps
          \item[PLAN:] 500 Mbps
          \item[WISH:] 1 Gbps
          \item[Mbps:] Megabits per second
          \item[Gbps:] Gigabits per second
        \end{itemize}

        Network latency

        \begin{itemize}
          \setlength{\itemindent}{.5in}
          \itemsep .15em
          \item[ID:] QR-0013
          \item[GIST:] TCP/IP network latency
          \item[SCALE:] microseconds
          \item[METER:] Measure the time difference between the time a message
            was sent and when it was received.
          \item[MUST:] 100 ms
          \item[PLAN:] 100 us
          \item[WISH:] 1 us
        \end{itemize}

        Data acquisition device speed

        \begin{itemize}
          \setlength{\itemindent}{.5in}
          \itemsep .15em
          \item[ID:] QR-0014
          \item[GIST:] Data acquisition device speed
          \item[SCALE:] S/s
          \item[METER:] Measurement obtained by counting the number of single
            measurement values acquired in 1 minute.
          \item[MUST:] 10 S/s
          \item[PLAN:] 1 kS/s
          \item[WISH:] 1 MS/s
          \item[s/s:] Samples per second
        \end{itemize}

        Hard drive read/write speed

        \begin{itemize}
          \setlength{\itemindent}{.5in}
          \itemsep .15em
          \item[ID:] QR-0015
          \item[GIST:] Hard drive transfer rate
          \item[SCALE:] Mbps
          \item[METER:] The transfer rate is given by the manufacturer.
          \item[MUST:] 100 Mbps
          \item[PLAN:] 500 Mbps
          \item[WISH:] More than 500 Mbps
        \end{itemize}

        Application memory usage

        \begin{itemize}
          \setlength{\itemindent}{.5in}
          \itemsep .15em
          \item[ID:] QR-0016
          \item[GIST:] The recommended RAM required to meet performance and user
            experience requirements.
          \item[SCALE:] GB
          \item[METER:] The quantity of the installed RAM.
          \item[MUST:] 1 GB
          \item[PLAN:] 2 GB
          \item[WISH:] 512 MB
          \item[GB:] Gigabytes
        \end{itemize}

        Application processor usage

        \begin{itemize}
          \setlength{\itemindent}{.5in}
          \itemsep .15em
          \item[ID:] QR-0017
          \item[GIST:] The CPU usage.
          \item[SCALE:] percent
          \item[METER:] Use the htop utility to measure this over a 1 hour time
            period.
          \item[MUST:] Less than 50\%
          \item[PLAN:] Less than 20\%
          \item[WISH:] Less than 10\%
        \end{itemize}

      \paragraph{Software System Attributes}

        Reliability

        \begin{itemize}
          \setlength{\itemindent}{.5in}
          \itemsep .15em
          \item[ID:] QR-0018
          \item[GIST:] Continuous operation.
          \item[SCALE:] days
          \item[METER:] Observe a full functioning system running continuously
            for a number of days.
          \item[MUST:] 90 days
          \item[PLAN:] 180 days
          \item[WISH:] 365 days
        \end{itemize}

        Availability

        \begin{itemize}
          \setlength{\itemindent}{.5in}
          \itemsep .15em
          \item[ID:] QR-0019
          \item[GIST:] System uptime.
          \item[SCALE:] percent
          \item[METER:] Observe a system as available for network requests as a
            ratio of time up to total time.
          \item[MUST:] 90\%
          \item[PLAN:] 99\%
          \item[WISH:] 99.99\%
        \end{itemize}

        \begin{itemize}
          \setlength{\itemindent}{.5in}
          \itemsep .15em
          \item[ID:] QR-0020
          \item[TITLE:] Network connection.
          \item[DESC:] The applications should be connected to a building
            network.
          \item[RAT:] In order for the applications to speak with each other.
          \item[DEP:] none
        \end{itemize}

        \begin{itemize}
          \setlength{\itemindent}{.5in}
          \itemsep .15em
          \item[ID:] QR-0021
          \item[TITLE:] Remote access.
          \item[DESC:] A user can access the system from outside the building
            network over a secure network connection.
          \item[RAT:] In order for a user to remotely monitor or download data
            from the system.
          \item[DEP:] none
        \end{itemize}

        \begin{itemize}
          \setlength{\itemindent}{.5in}
          \itemsep .15em
          \item[ID:] QR-0021
          \item[TITLE:] Control access.
          \item[DESC:] The daemons should provide a means to enable or disable
            secure communications through an authentication mechanism.
          \item[RAT:] It should be possible to prevent unauthenticated requests.
          \item[DEP:] none
        \end{itemize}

        Maintainability

        \begin{itemize}
          \setlength{\itemindent}{.5in}
          \itemsep .15em
          \item[ID:] QR-0022
          \item[TITLE:] Software upgrades.
          \item[DESC:] Bug fixes and software upgrades can be applied using a
            package manager.
          \item[RAT:] Maintaining the software should be made simple by using
            an automated system.
          \item[DEP:] none
        \end{itemize}

    %\subsubsection{Core Library}\label{sec:req-srs-core}

      %\paragraph{Specific Requirements}

\paragraph{Functional Requirements}

\paragraph{Non-functional Requirements}


    %\subsubsection{Data Acquisition Library}\label{sec:req-srs-daq}

    %\subsubsection{Data Logging Library}\label{sec:req-srs-log}

    %\subsubsection{Process Control Library}\label{sec:req-srs-ctl}

    %\subsubsection{Services/Networking Library}\label{sec:req-srs-net}

    %\subsubsection{Device Plugins}\label{sec:req-srs-dev-plug}

    %\subsubsection{Logging Plugins}\label{sec:req-srs-log-plug}

    %\subsubsection{Control Plugins}\label{sec:req-srs-ctl-plug}

    %\subsubsection{Data Acquisition Service}\label{sec:req-srs-daq}

    %\subsubsection{Data Logging Service}\label{sec:req-srs-log}

    %\subsubsection{Process Control Service}\label{sec:req-srs-control}
