\section{Installation}\label{sec:inst}

  \subsection{Environment}\label{sec:inst-env}

    OpenDCS and it's constituent components have been designed to operate
    within an environment that is built of or provides the following:

    \begin{itemize}
      \item GNU/Linux
      \item Automake/Autoconf
      \item GLib/GNOME
      \item LAN connection
    \end{itemize}

    The intention is to be able to install and run the project libraries and
    executables on any modern Linux installation. Fedora Workstation has been
    the dominant development environment but with minor modification what's
    given in this document should work on any number of distributions including
    Ubuntu 14.04 and newer, Debian 7 and newer, Arch, and CentOS 7.

  \subsection{Compilation}\label{sec:inst-comp}

    \begin{lstlisting}[caption={Compiling OpenDCS},
                       label={lst:inst-comp}]
			dcs 0.2.0

			Options

			Prefix: ...................................... : /usr/local
			Libdir: ...................................... : ${exec_prefix}/lib
			Optimized Build: ............................. : yes
			WebKit: ...................................... : yes

			Development Options

			DAQ support: ................................. :
			UI support: .................................. : yes
			Enable Debug: ................................ : yes
			Enable Tracing: .............................. : no
			Enable Profiling (-pg): ...................... : yes
			Build Test Suite: ............................ : yes
			Build documentation: ......................... : yes
			Build API reference: ......................... : yes
			Update MIME Database: ........................ : no

			Plugins

			(placeholder): ............................... :

			Backends

			XML: ......................................... : yes

			Devices

			Arduino: ..................................... : yes
			Comedi: ...................................... : yes
			MCC USB: ..................................... : yes

			Control Loops

			PID: ......................................... : yes

			Example Plugins

			Example Python: .............................. : yes
			Example Vala: ................................ : yes

			Languages support

			Python 3 support: ............................ : no

			dcs will be installed in ${exec_prefix}/bin

			Now type `make' to compile
    \end{lstlisting}

    \subsubsection{Controlling Installation}\label{sec:inst-comp-control}

      % TODO complete this section
      Currently, if OpenDCS was installed from source using the instructions
      given in this document the installation path is relative to
      \texttt{/usr/local}, it is possible to change this behaviour using the
      build system to include an installation prefix using
      \texttt{./autogen.sh --prefix=/opt} for example. This will install
      the libraries and \ldots

  \subsection{Dependencies}\label{sec:inst-dep}

    The commands for installing the software dependencies that are required to
    be installed to fully compile all components of this project, including
    those needed to build the UI, are given in Listing~\ref{lst:inst-dep} for
    Fedora using release 19 or newer.  While this project has been tested using
    other variants of Linux, namely Ubuntu and Arch, Fedora has been the
    primary development environment.

    These instructions include building and installing libcld, while this
    library is no longer an explicit dependency of OpenDCS as described in this
    document it is included due to the existence of some components that have
    been intentionally omitted in order to be able to compile.

    \begin{lstlisting}[language=bash,
                       caption={Installation Dependencies},
                       label={lst:inst-dep}]
      # Install packages required
      sudo dnf install automake autoconf libtool gnome-common intltool gcc vala
      sudo dnf install glib2-devel gtk3-devel libxml2-devel libgee-devel \
       json-glib-devel clutter-devel clutter-gtk-devel gsl-devel \
       gtksourceview3-devel libmatheval-devel sqlite-devel \
       gobject-introspection-devel gettext-devel gettext-common-devel \
       libmodbus-devel comedilib-devel librsvg2-devel python3-devel \
       pygobject3-devel libpeas-devel libsoup-devel webkitgtk4-devel \
       json-glib-devel libpeas-devel zeromq3-devel
      # Install Vala dependencies
      git clone https://github.com/geoffjay/modbus-vapi.git
      git clone https://github.com/geoffjay/comedi-vapi.git
      sudo mkdir -p /usr/local/lib/pkgconfig
      sudo cp comedi-vapi/comedi.pc /usr/local/lib/pkgconfig/
      ver=`vala --version | sed -e 's/.*\([0-9]\.[0-9][0-9]\).*/\1/'`
      sudo cp comedi-vapi/comedi.vapi /usr/share/vala-$ver/vapi/
      sudo cp modbus-vapi/libmodbus.vapi /usr/share/vala-$ver/vapi/
      # Install transitional library dependancy
      git clone https://github.com/geoffjay/libcld.git
      cd libcld
      git checkout master
      export PKG_CONFIG_PATH=/usr/local/lib/pkgconfig
      ./autogen.sh
      make && sudo make install
      cd ..
      echo "/usr/local/lib" | sudo tee --append /etc/ld.so.conf
      sudo ldconfig
    \end{lstlisting}

  \subsection{Containerized Components}\label{sec:inst-cont}

    \begin{lstlisting}[caption={Building Containers},
                       label={lst:inst-cont}]
    \end{lstlisting}

  \subsection{Plugins}\label{sec:inst-plugins}

    While it is possible for plugin systems that have been developed using
    libpeas to load plugins from any path on the system that functionality was
    not added, though it likely will be in a future version. Using the default
    installation prefix of \texttt{/usr/local} the services will search the
    path \texttt{/usr/local/lib/dcs/devices} for DAQ device plugins,
    \texttt{/usr/local/lib/dcs/backends} for data logging backend plugins,
    and \texttt{/usr/local/lib/dcs/controllers} for feedback control plugins.

    Plugins are loaded if there is an associated plugin information file in the
    paths given above. They describe various pieces of the plugin that they
    system requires to load it, as well as document who to blame for their
    efforts. An example of the plugin that is included in OpenDCS is given in
    Listing~\ref{lst:inst-plugin-info}.

    \begin{lstlisting}[caption={Plugin Information File Example},
                       label={lst:inst-plugin-info}]
      [Plugin]
      Module=signal-generator
      Loader=python3
      Name=Signal Generator Device Plugin
      Description=Signal generator device.
      Authors=Geoff Johnson <geoff.jay@gmail.com>
      Copyright=Copyright © 2016 Geoff Johnson
    \end{lstlisting}

    Plugins can be developed using C, Vala, and Python. A minimal working
    example, that does nothing more that loads and prints a message on
    activation and deactivation, has been provided in
    Listing~\ref{lst:inst-plugin-example}.

    \begin{lstlisting}[caption={Minimal Working Example of a libpeas Plugin},
                       label={lst:inst-plugin-example}]
			#
			# Sample OpenDCS python plugin.
			#
			# Taken from: https://github.com/gregier/libpeas/tree/master/peas-demo
			#
			# For gi to find dcs this needs to be run before starting the application:
			#   export GI_TYPELIB_PATH=/usr/local/lib/girepository-1.0/:$GI_TYPELIB_PATH

			import gi
			gi.require_version('Peas', '1.0')
			gi.require_version('DcsCore', '0.2')
			gi.require_version('DcsDAQ', '0.2')
			gi.require_version('DcsNet', '0.2')
			from gi.repository import GObject
			from gi.repository import Peas
			from gi.repository import DcsCore
			from gi.repository import DcsDAQ
			from gi.repository import DcsNet

			from pprint import pprint

			LABEL_STRING="Signal Generator Device"

			class SignalGenerator(DcsDAQ.DAQDevice):
					__gtype_name__ = 'SignalGenerator'

					object = GObject.property(type=GObject.Object)

					def do_activate(self):
							print("SignalGenerator.do_activate")

					def do_deactivate(self):
							print("SignalGenerator.do_deactivate")

					def do_update_state(self):
							print("SignalGenerator.do_update_state")
    \end{lstlisting}
