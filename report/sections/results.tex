\section{Results}\label{sec:results}

  Section 7.2 of the proposal document submitted to BCIT CST BTech outlined all
  tasks that were expected to be achieved as a requirement of this project were
  carried out with a single exception. Part of Task 0.1.8, ``Write simple
  example program as proof-of-concept'' was not. It was felt that the
  justification for developing such a tool was to validate the network
  connectivity and in general the interoperability of this project, but these
  were accomplished using existing tools, namely the command \textt{zmqc} and
  Postman so the time and effort was placed on other aspects. For convenience
  that proposal has been included as a soft copy with this report submission at
  the location \texttt{documents/proposal.pdf}.

  \subsection{Successes}\label{sec:results-successes}

    There were a couple of very important components of this project that
    needed to be carried out successfully, the ability to configure the
    services using a common and consistent format, that each system service
    could load context specific plugins that would perform the heavy lifting
    for data acquisition, data logging, and feedback control, and REST APIs
      which would expose the services and data gathered to just about anything
      that consumes HTTP traffic.

    The configuration systems, which not perfect, do provide the ability to
    construct networked services with a varying number of socket types and
    protocol methods at runtime without writing any additional code, and
    without even needing to compile any code. This may exist elsewhere in
    similar forms, but not that this author has been able to find so the
    ability to perform this kind of flexible networking is seen as a project
    accomplishment.

    Plugin systems are crucial to this system, the project was to define a
    framework for application development that use a variety of different kinds
    of data acquisition hardware devices, data logging backends, and feedback
    control schemes. This was achieved by making each service load plugins and
    expose parts of their data models to those plugins so that the actual work
    could be done without knowing what the hardware is or need to do for the
    development of the services.

    REST service routers open up a lot of possibilities including websites that
    pull data using standard web requests, this should make application
    development much easier long term for any project that needs to pull data
    from a data log, retrieve a measured value, or perform complicated control
    systems without having to actual modify any of the systems that perform
    those tasks.

  \subsection{Failures}\label{sec:results-failures}

    The idea of being able to have services and plugins be configurable from a
    variety of input types was an interesting one as it would allow for the
    developer of such things to make choices about how they wanted to configure
    their component, but it created added complexity where it wasn't necessary.
    Using a single format, ideally JSON, would have simplified development and
    much of the development could have been focussed on systems that were more
    crucial to execution.

    Configuration exceptions are too restrictive and not always handled
    correctly so the service will fail if there are issues with what's been
    provided, such as an object that the service does not have a factory to
    construct. This is not a show stopper due to the fact that the editor
    shouldn't need to add objects that aren't available in the libraries, but
    it should skip them gracefully with the addition of a warning message and
    allow the rest of the data model to be constructed successfully.
